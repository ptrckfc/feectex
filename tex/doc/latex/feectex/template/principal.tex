%% principal.tex
%!TeX encoding = UTF-8

% Define o grau acadêmico e a língua principal do documento.
% ATENÇÃO: a última língua declarada é a língua principal do
% documento.
\documentclass[
    doutorado,              % Selecionar grau acadêmico: mestrado ou doutorado.
%    cotutela,              % Remover comentário se houver cotutela.
%    enlgish,
%    french,
%    spanish,
    brazil
]{feectex}

% Pacotes adicionais (adicione ou remova conforme sua
% necessidade). Visite o site https://www.ctan.org para
% consultar outros pacotes.
\usepackage{amsmath, amsfonts, amssymb}
\usepackage{pdfpages}       % Permite importar arquivos PDF.
\usepackage{hyperref}       % Viabiliza o uso de links internos e externos.
\usepackage{lmodern}        % Emprega a fonte Latin Modern para o texto.
\usepackage{subcaption}     % Inclusão de múltiplas figuras.

% Diretório onde se encontram as imagens utilizadas no trabalho.
% O comando pode ser modificado para incluir diversos diretórios:
%   \graphicspath{{dir1/}{dir2/}...}
\graphicspath{{./figuras/}}

%
% Informações da obra.
%

% Qual é o título da obra?
\titulo{Título da Obra}

% Qual é o seu nome e o número do seu RA?
\autora{Nome da Autora}
%\autor{Nome do Autor}
\ra{999999}

% Qual é o título e o nome da sua orientadora ou orientador?
\orientadora{Profa. Dra. Nome da Orientadora}
%\orientador{Prof. Dr. Nome do Orientador}

% Qual é o título e o nome da sua coorientadora ou coorientador?
% Se não houver, comente ambas as linhas abaixo.
% \coorientadora{Profa. Dra. Nome da Coorientadora}
\coorientador{Prof. Dr. Nome do Coorientador}

% Qual é a universidade e o instituto ou faculdade?
\universidade{Universidade Estadual de Campinas}
\institutooufaculdade{Faculdade de Engenharia Elétrica e de Computação}

% Qual é o nome da universidade de cotutela? Se houver,
% modificar a opção na declaração da classe acima
% (\documentclass(...)). Caso contrário, comente a linha abaixo.
% \universidadecotutela{Nome da Universidade (País)}

% Qual é a cidade onde foi ou será realizada a defesa?
\local{Campinas}

% Qual data foi ou será realizada a defesa?
\diadefesa{XX}
\mesdefesa{janeiro}
\anodefesa{20XX}

% Qual é a área de concentração do trabalho? Selecione uma.
% Estas são as áreas de concentração informadas no Catálogo
% dos Cursos de Pós-Graduação Strictu Sensu - UNICAMP -
% 2023, no Programa de Engenharia Elétrica.
% \areaconcentracao{Automação}
% \areaconcentracao{Eletrônica, Microeletrônica e Optoeletrônica}
% \areaconcentracao{Engenharia Biomédica}
\areaconcentracao{Engenharia de Computação}
% \areaconcentracao{Energia Elétrica}
% \areaconcentracao{Telecomunicações e Telemática}

% Quais são os títulos e os nomes dos componentes da banca
% examinadora? (Por enquanto, é necessário repetir o título
% e o nome da orientadora ou orientador aqui.)
\bancaexaminadora{%
    Prof. Dr. Nome do Orientador (Presidente) \\%
    Prof. Dr. Nome do Primeiro Membro \\%
    Prof. Dr. Nome do Segundo Membro \\%
    Prof. Dr. Nome do Terceiro Membro \\%
    Prof. Dr. Nome do Quarto Membro
}

\begin{document}
    % \hyphenation{}

    % Elementos pré-textuais:
    %   capa;
    %   folha de rosto;
    %   ficha catalográfica;
    %   folha de aprovação;
    %   dedicatoria;
    %   agradecimentos;
    %   epigrafe;
    %   resumo / abstract;
    %   lista de ilustrações;
    %   lista de tabelas;
    %   lista de abreviaturas e siglas;
    %   lista de simbolos;
    %   sumário.
    % Comente os elementos não utilizados no seu trabalho.
    \imprimircapa
    \imprimirfolhaderosto
    \imprimirfichacatalografica{ficha-catalografica.pdf}
    \imprimirfolhaaprovacao
    \imprimirdedicatoria{dedicatoria}
    \imprimiragradecimentos{agradecimentos}
    \imprimirepigrafe{epigrafe}
    \imprimirresumo{resumo}
    \imprimirlistailustracoes
    \imprimirlistatabelas
    \imprimirlistaabreviaturassiglassimbolos
    \imprimirsumario

    % Elementos textuais.
    \textual
    \part{Primeira Parte}
    %!TeX root = ../principal.tex
%!TeX encoding = utf-8
\chapter{Primeiro Capítulo}
\label{cap1}

Comece a redigir o seu texto a partir daqui.

Para teste das citações: \citar{Turing:1937, Shannon:1948}.


    % Elementos pós-textuais:
    %   bibliografia / referências;
    %   apêndices;
    %   asnexos.
    % Comente os elementos não utilizados no seu trabalho.
    \postextual
    \bibliography{bibliografia}
    \apendices
    %!TeX root = ../principal.tex
%!TeX encoding = utf-8
\chapter{Lorem Ipsum}
\label{apendice}

Texto do apêndice.

    \anexos
    %!TeX encoding = UTF-8
%!TeX root = ../principal.tex
\chapter{Aliquam Dignissim}

Texto do anexo.

\end{document}
